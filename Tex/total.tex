\chapter*{非平衡量子场论读书报告(张迎姬 2024.5)}
本篇读书报告主要参考 RMP 86,779(2014) 和 RMP 68,13(1996) 两篇综述,从平衡态理论出发,简要介绍非平衡 DMFT 的理论框架。

在处理晶格系统的电子相互作用时,最简单的方法是做平均场近似。但这种方法得到的是静态的库仑势,电子因此只能形成静态的长程序,时间关联的电子涨落无法得到很好描述,因此人们无法得到例如 Mott 绝缘性相关的强关联物理图像。为了处理强关联电子体系,人们发展了动力学平均场理论(Dynamical Mean Field Theory, DMFT)。DMFT 方法处理的不是电子密度,而是与频率依赖的格林函数 $G(\omega)$,因此它能够自然地将时间涨落相关信息包括进来。利用 DMFT 方法的这一优势,并进一步将这套技术推广到非平衡体系,人们就能够在热力学极限下直接处理时间域的量子涨落。在此框架下,人们更好地理解许多非平衡体系的物理性质,例如莫特绝缘体中的介电击穿,光掺杂以及淬火系统中的崩溃和复苏振荡(collapse-and-revival oscillations)等。

DMFT 的核心思想是将晶格模型自洽地映射为单杂质模型,从而把物理问题等效地转化成对单杂质的求解问题。非平衡 DMFT 是对平衡态理论框架的一个直接推广,利用双时格林函数和相应的时间依赖的 Weiss 场自洽地描述系统的有效作用量。下面从平衡态出发,对这一方法做简要推导。

为了将晶格模型的性质映射到单杂质模型中,一个直接的方式是扣除晶格模型的一个格点,利用路径积分将其他格点积分掉,得到与单杂质模型数学形式相同的有效作用量。文献中一般称这种推导方法为空腔法(cavity method)。

Hubbard 模型的哈密顿量为
\begin{equation}
    \hat{H}=-\sum_{ij\sigma}t_{ij}^{ }c_{i\sigma}^\dg c_{j\sigma}^{ }+U\sum_i n_{i\ua} n_{i\da},
\end{equation}
其中, 设 $t_{ii}=0$, $t_{ij}=t_{ji}\in \mathbb{R}$。

在相干态路径积分下, 有配分函数 
\begin{equation}
    Z=\int\prod_{i}\prod_\sigma \mathcal{D}c_{i\sigma}^*(\tau)\mathcal{D}c_{j\sigma}^{ }(\tau)\e^{-S},
\end{equation}
其中作用量的表达式为
\begin{equation}
    S=\int_0^\infty\dd\tau\lmi \sum_{i\sigma}c_{i\sigma}^*(\tau)\ls \frac{\pt}{\pt \tau}-\mu \rs c_{i\sigma}(\tau)-\sum_{ij}\sum_\sigma t_{ij}c_{i\sigma}^*(\tau)c_{j\sigma}^{ }(\tau)+U\sum_i n_{i\ua}(\tau)n_{i\da}(\tau) \rmi,
\end{equation}
将作用量划分为空腔 $o$ 格点项 $S_o$, 剩余格点项 $S^{(o)}$ 以及相互作用项 $\Delta S$, 其中 
\begin{equation}
    S_o=\int_0^\beta\dd\tau\lmi \sum_\sigma c_{o\sigma}^*(\tau)\ls \frac{\pt}{\pt \tau}-\mu \rs c_{o\sigma}^{ }(\tau)+Un_{o\ua}(\tau)n_{o\da}(\tau) \rmi,
\end{equation}
\begin{equation}
    \Delta S=-\int_0^\beta \dd\tau\sum_{i\sigma}t_{io}\lmi c_{i\sigma}^*(\tau)c_{o\sigma}^{ }(\tau)+c_{o\sigma}^*(\tau)c_{i\sigma}^{ }(\tau) \rmi.
\end{equation}
记 $\eta_{i\sigma}(\tau)\equiv t_{io}c_{o\sigma}(\tau)$, 并对 $t_{ij}$ 作标度变换:
\begin{equation}
    t_{ij}=\frac{\tilde{t}_{ij}}{D^{\frac{|i-j|}{2}}}.
\end{equation}
D. Vollhardt 证明, 这样的取法可以在大维度极限 $D\to \infty$ 时得到有限的 $\lag H_T \rag$, 否则将发散或趋于零\cite{PhysRevLett.62.324}。

定义空腔格点的有效作用量:
\begin{equation}
    \frac{1}{Z_{\fun{eff}}}\e^{-S_{\fun{eff}}}\equiv \frac{1}{Z}\int\prod_{i\ne o}\prod_{\sigma}\mathcal{D}c_{i\sigma}^*\mathcal{D}c_{i\sigma}^{ }\e^{-S},
\end{equation}
为方便表述, 令 $\eta_1^*,\cdots,\eta_{N}^*\equiv \eta_{N+1},\cdots,\eta_{2N}$, 视为和 $\eta_1,\cdots,\eta_{N}$ 相互独立的变量。利用 Grassmann 函数的泰勒展开式:
\begin{equation}
    \begin{aligned}
        S_{\fun{eff}}(\eta_{1},\cdots,\eta_{2N})=&\int\dd\tau L_{\fun{eff}}\\
        =&\int\dd\tau\sum_{n=0}^\infty\sum_{i_1\cdots i_n=1}^{2N}\frac{1}{n!}\left.\frac{\pt^n L_{\fun{eff}}}{\pt\eta_{i_1}\cdots\pt \eta_{i_n}}\right|_{\{\eta_1,\cdots,\eta_{2N}=0\}}\eta_{i_n}\cdots\eta_{i_1},
    \end{aligned}
\end{equation}
利用 
\begin{equation}
    \frac{\pt \eta_{i\sigma}(\tau)}{\pt \eta_{j\sigma'}(\tau')}=\delta_{ij}\delta_{\sigma\sigma'}\delta(\tau-\tau')
\end{equation}
以及有效作用量的定义式, 可以计算各阶偏导数的表达式:
\begin{equation}
    \begin{aligned}
        -\e^{-S_{\fun{eff}}}\int\dd\tau\frac{\pt L_{\fun{eff}}}{\pt \eta_{i_1\sigma}}&=\int\prod_{k\ne o}\prod_\sigma\mathcal{D}c_{k\sigma}^*\mathcal{D}c_{k\sigma}^{ }\e^{-S}\lmi -\int\dd\tau c_{i_1\sigma}^*\rmi\\
        \Rightarrow \int\dd\tau\lmi \e^{-S_{\fun{eff}}}\frac{\pt L_{\fun{eff}}}{\pt \eta_{i_1}} \rmi &=\int\dd\tau\lmi \int\prod_{k\ne o}\prod_\sigma\mathcal{D}c_{k\sigma}^*\mathcal{D}c_{k\sigma}^{ }\e^{-S} c_{i_1\sigma}^* \rmi,
    \end{aligned}
\end{equation}
从中得到 
\begin{equation}
    \begin{aligned}
        \left.\frac{\pt L_{\fun{eff}}}{\pt \eta_{i_1}}\right|_{\{\eta_{1,\cdots,N},\eta_{1,\cdots,N}^*=0\}}=&\frac{\int\prod_{k\ne o}\prod_{\sigma}\mathcal{D}c_{k\sigma}^*\mathcal{D} c_{k\sigma}^{ }\e^{-S^{(o)}}c_{i_1\sigma}^*}{\int\prod_{k\ne o}\prod_{\sigma}\mathcal{D}c_{k\sigma}^*\mathcal{D} c_{k\sigma}^{ }\e^{-S^{(o)}}}\\
        =&\lag c^*_{i_1\sigma}(\tau_{i_1}) \rag^{(o)},
    \end{aligned}
\end{equation}
但是奇数阶 Grassmann 变量的期望值都是 0, 所以这里得到的一阶导为零。再求一次导, 得到 
\begin{equation}
    \begin{aligned}
        \left.\frac{\pt^2 L_{\fun{eff}}}{\pt \eta_{i_1}^{ }\pt \eta_{j_1}^*}\right|_{\{\eta_{1,\cdots,N},\eta_{1,\cdots,N}^*=0\}}&=-\left.\frac{\pt^2 L_{\fun{eff}}}{\pt \eta_{i_1}^{*}\pt \eta_{j_1}^{ }}\right|_{\{\eta_{1,\cdots,N},\eta_{1,\cdots,N}^*=0\}}\\
        &=- \lag c_{j_1\sigma}^{ }(\tau_{j_1})c_{i_1\sigma}^*(\tau_{i_1}) \rag^{(o)}\\
        &=-G^{(o)}(\tau_{j_1},\tau_{i_1}).
    \end{aligned}
\end{equation}
同理, 应有
\begin{equation}
    \left.\frac{\pt^{2n}L_{\fun{eff}}}{\pt\eta_{j_1\sigma}^*\cdots\pt\eta_{j_n}^*\pt \eta_{i_1}^{ }\cdots\pt \eta_{i_n}^{ }}\right|_{\{\eta_{1,\cdots,N},\eta_{1,\cdots,N}^*=0\}}=(-1)^nG^{(o)}(\tau_{j_1}\cdots\tau_{j_n},\tau_{i_1}\cdots\tau_{i_n}).
\end{equation}
在 Taylor 展开式中, 首先, 只考虑 $\eta\eta^*$ 耦合, 不考虑 $\eta\eta$ 和 $\eta^*\eta$ 耦合, 其次, $2n$ 个 Grassmann 变量换序时会产生符号变化 $(-1)^n$ 与上式的符号抵消, 最后, 对应每一阶 $n$, 确定了 $i_1,\cdots,i_n,j_1,\cdots,j_n$ 的取值后, 应有 $n!$ 种排列组合方式, 写成正规形式后可以合并为一项。这样, 展开式可以表示为
\begin{equation}
    \begin{aligned}
        S_{\fun{eff}}=&\sum_{n=0}^\infty \sum_{\lag i_1,\cdots,i_n=1 \rag}^N\sum_{\lag j_1,\cdots,j_n=1 \rag}^N\int \left.\frac{\pt^{2n} L_{\fun{eff}}}{\pt \eta_{i_1}^{ }\cdots\pt\eta_{i_n}^{ }\pt \eta_{j_1}^*\cdots\pt\eta_{j_n}^*}\right|_{\{\eta_1^{ },\cdots,\eta_N^*=0\}}\eta_{j_n}^*\cdots\eta_{j_1}^*\eta_{i_n}\cdots\eta_{i_1}\\
        =&\sum_{n=1}^\infty \sum_{i_1\cdots j_n}\int\eta_{i_1}^*(\tau_{i_1})\cdots \eta_{i_n}^*(\tau_{i_n})\eta_{j_1}(\tau_{j_1})\cdots \eta_{j_n}(\tau_{j_n})G_{i_1\cdots j_n}^{(o)}(\tau_{i_1}\cdots \tau_{i_n},\tau_{j_1}\cdots \tau_{j_n})+S_o+\fun{const.}
    \end{aligned}
\end{equation} 
其中, 忽略了 $\eta_{i\sigma}(\tau)$ 的自旋指标和对 $\tau$ 的依赖关系。此即综述中 (34) 式。在大维度极限 $D\to \infty$ 下, 可以证明, 各阶展开中, 只有 $n=1$ 项保留下来, 其他高阶项快速衰减。于是有效作用量可以简化为 
 \begin{equation}
    S_{\fun{eff}}=S_o+\int_0^\beta\dd\tau\int_0^\beta\dd\tau'\sum_\sigma c_{i\sigma}^*(\tau)\lmi \sum_{ij}t_{oi}t_{oj}G_{ij}^{(o)}(\tau-\tau') \rmi c_{o\sigma}^{ }(\tau'),
 \end{equation}
中括号内的表达式被称为动力学平均场。整个积分式描写了周围格点对 $o$ 格点的影响。另外, $S_o$ 的表达式为 
\begin{equation}
    S_o=\int_0^\beta\dd\tau\sum_\sigma\lmi c_{o\sigma}^*(\tau)\ls \frac{\pt}{\pt \tau}-\mu \rs c_{o\sigma}^{ }(\tau) \rmi+\int_0^\beta\dd\tau\lmi U n_{o\ua}(\tau)n_{o\da}(\tau) \rmi.
\end{equation} 
由单杂质安德森模型 
\begin{equation}
    H_{\fun{SAIM}}=\epsilon_d\sum_\sigma d_\sigma^\dg d_\sigma^{ }+\sum_{k\sigma}\lmi \epsilon_k c_{k\sigma}^{\dg} c_{k\sigma}^{ }+V_k\ls d_\sigma^\dg c_{k\sigma}^{ }+c_{k\sigma}^\dg d_\sigma^{ } \rs \rmi+U n_{\ua}^d n_{\da}^d, 
\end{equation}
对应的杂质作用量为 
\begin{equation}
    S_{\fun{imp}}[d_\sigma^*,d_\sigma^{ }]=\int_0^\beta\dd\tau\int_0^\beta\dd\tau'd_\sigma^*(\tau)\lmi -\mathcal{G}^{-1}_0(\tau-\tau') \rmi d_{\sigma}(\tau')+U\int_0^\beta\dd\tau n_{\ua}^d(\tau)n_{\da}^d(\tau).
\end{equation}
上面得到的 $S_{\fun{eff}}$ 可以写成类似的形式, 即
\begin{equation}
    \begin{aligned}
        S_{\fun{eff}}=&\int_0^\beta\dd\tau \int_0^\beta\dd\tau'\sum_\sigma c_{o\sigma}^*(\tau)\lmi \ls \frac{\pt}{\pt \tau}-\mu \rs\delta(\tau-\tau')+\sum_{ij}t_{oi}t_{oj}G_{ij}^{(o)}(\tau-\tau') \rmi c_{o\sigma}(\tau')\\
        & +\int_0^\beta\dd\tau Un_{o\ua}(\tau)n_{o\da}(\tau)\\
        =&\int_0^\beta\dd\tau \int_0^\beta\dd\tau'\sum_\sigma c_{o\sigma}^*(\tau)\lmi \delta^{(1)}(\tau-\tau')-\mu\delta(\tau-\tau') +\sum_{ij}t_{oi}t_{oj}G_{ij}^{(o)}(\tau-\tau') \rmi c_{o\sigma}(\tau')\\
        &+\int_0^\beta\dd\tau Un_{o\ua}(\tau)n_{o\da}(\tau),
    \end{aligned}
\end{equation}
对比两式, 得到 
\begin{equation}
    \mathcal{G}_0^{-1}(\tau-\tau')=\delta^{(1)}(\tau-\tau')-\mu\delta(\tau-\tau') +\sum_{ij}t_{oi}t_{oj}G_{ij}^{(o)}(\tau-\tau'),
\end{equation}
人们一般也将这里的无库仑相互作用格林函数 $\mathcal{G}_0$ 称为 Weiss 场。这样就将晶格模型映射为了杂质模型。

在一般情况下,格点自能应该是动量依赖的,但在无穷维极限下,格点自能是对角的。在这种近似下,格点自能函数就等于空腔的自能函数,没有动量依赖\cite{Hartmann1989}。此外,为了自洽地将晶格模型映射到杂质模型,要求两个体系的格点可观测量和相互作用项都对应相等,也就是要求两个体系的自能函数相同,因此就得到杂质自能等于空腔自能,即 
\begin{equation}
    \Sigma_{\fun{imp}}(\tau-\tau')=\Sigma_{oo}(\tau-\tau').\label{dmftscf1}
\end{equation}
上式等价于
\begin{equation}
    G_{\fun{imp}}(\tau-\tau')=G_{oo}(\tau-\tau').\label{dmftscf2}
\end{equation}

利用杂质模型的 Dyson 方程,可以得到 
\begin{equation}
    G_{\fun{imp}}(\ii\omega_n)=\lb \mathcal{G}_0^{-1}(\ii\omega_n)-\Sigma_{\fun{imp}}(\ii\omega_n) \rb^{-1},\label{dmftscf3}
\end{equation}

另外,在 $D=\infty$ 极限下,晶格模型的格林函数具有下面的关系式\cite{PhysRevB.77.235106} 
\begin{equation}
    G_{ij}^{(o)}(\ii\omega_n)=G_{ij}(\ii\omega_n)-\frac{G_{io}(\ii\omega_n)G_{oj}(\ii\omega_n)}{G_{oo}(\ii\omega_n)},\label{lattice-G}
\end{equation}

以上四式已构成封闭方程组。在实际计算中,人们还需要作进一步简化处理,在此暂不加赘述。

以上是在平衡态体系下推导出的DMFT自洽方程。非平衡体系的 DMFT 可以看作在此基础上的一个推广,形式上与平衡态类似,需要将描述平衡态粒子涨落的 Weiss 场推广为依赖于两个时间指标的含时 Weiss 场 $\Delta(t,t')$,从而定义有效相互作用对应的双时格林函数。考虑到积分围道总是具有有限的长度,人们也可以对围道序的格林函数作类似于平衡态的标度变换。这样就能够得到非平衡 DMFT 的杂质自能等于空腔自能函数 
\begin{equation}
    \Sigma_{\fun{imp}}(t, t')=\Sigma_{oo}(t,t'),
\end{equation}
围道编序的格林函数形式为 $G_{ij}(t,t')=-\ii \lag \mathcal{T}_C c_{i\sigma}^{ }(t)c_{j\sigma}^\dg (t') \rag$,可以通过如下形式的戴森方程得到:
\begin{equation}
    \ls G^{-1} \rs_{ij}(t,t')=\lmi \delta_{ij}(\ii\pt_t+\mu)-v_{ij}(t) \rmi \delta_C(t,t')-\delta_{ij}\Sigma_{ii}(t,t').
\end{equation}
在 DMFT 的 $d\to \infty$ 惯用伎俩下,为了得到正确形式的 $\Sigma_{ii}[G]$,人们需要求解一般形式的杂质模型作用量:
\begin{equation}
    S_i=-\ii\int_C\dd tH_{\fun{loc}}(t)-\ii\sum_\sigma\int_C\dd t\dd t' c_\sigma^\dg (t)\Delta_i(t,t')c_\sigma^{ }(t'),
\end{equation}
式中 $\Delta(t,t')$ 须保证格林函数满足 
\begin{equation}
    G_{ii}(t,t')=-\ii\lag \mathcal{T}_C c(t)c^\dg (t') \rag_{Si},
\end{equation}
相应的自能函数由戴森方程定义:
\begin{equation}
    G_{ii}^{-1}(t,t')=(\ii\pt_t+\mu)\delta_C(t,t')-\Sigma_{ii}(t,t')-\Delta_i(t,t').
\end{equation}
与平衡态的推导类似,以上五式已构成非平衡 DMFT 封闭的自洽方程组。看起来这些方程的数学形式与平衡态类似,但非平衡框架下的 DMFT 自洽方程组具有完全不同的物理意义。这里自能函数对应的戴森方程可以看作是格林函数 $G_{ii}$ 的非马尔可夫动力学方程,所以实际上非平衡 DMFT 方程组是一个非线性初值问题。这一套工具可以应用在很多实际物理问题中,例如任意形式的电磁驱动场,耗散或非耗散系统,以及各种不同类型的局域相互作用系统。

在实际材料问题中的强关联电子结构计算里,人们广泛应用的方法是将第一性原理计算和 DMFT 相结合,这样人们能够有效定量分析时间分辨的光电子谱等可观测量。而这些应用又对方法本身提出了进一步要求,例如人们需要考虑如何在非平衡态下将电子结构 downfold 到有效的格点模型中,以及如何处理远离线性响应区间的电磁场等等。